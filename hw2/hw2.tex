\documentclass[11pt]{article}

% Packages
\usepackage[utf8]{inputenc}
\usepackage{amsmath, amssymb, amsthm}
\usepackage{enumitem}
\usepackage{geometry}
\usepackage{fancyhdr}
\usepackage{mathtools}
\usepackage{multicol}

% Page layout
\geometry{margin=0.75in}
\pagestyle{fancy}
\fancyhf{}
\rhead{Ian Gallagher}
\lhead{Math 207C Homework}
\rfoot{\thepage}
\setlength{\headheight}{14pt}

% Commands
\newcommand{\vep}{\varepsilon}
\DeclarePairedDelimiter\abs{\lvert}{\rvert}

% Environments
\newtheoremstyle{problemstyle}
  {1em} % Space above
  {1em} % Space below
  {\normalfont} % Body font
  {} % Indent amount
  {\bfseries} % Theorem head font
  {} % Punctuation after theorem head
  {\newline} % Space after theorem head
  {} % Theorem head spec

\theoremstyle{problemstyle}
\newtheorem{problem}{Problem}

% Custom commands
\newenvironment{solution}
  {\noindent\textbf{Solution}\quad}
  {\hfill$\blacksquare$\par\vspace{1em}}

% Enumerate styles
\setlist[enumerate,1]{label=(\roman*), ref=\roman*}
\setlist[enumerate,2]{label=(\alph*), ref=\alph*}

% Title info
\title{Math 207C Homework \#\texttt{2}}
\author{Ian Gallagher}
\date{\today}

\begin{document}

\maketitle

\begin{problem}
  Holmes \# 1.8, 1.9, 1.16
  \begin{enumerate}
    \item[(1.8)] Find the first two terms in the expansion for the following
      functions:
      \begin{enumerate}
          \item \( f = \displaystyle\int_0^{\pi/4}
            \frac{dx}{\varepsilon^2 + \sin^2 x}. \)
          \item \( f = \displaystyle\int_0^1 \frac{\cos(\varepsilon
            x)}{\varepsilon + x} \, dx. \)
          \item \( f = \displaystyle\int_0^1 \frac{dx}{\varepsilon +
            x(x - 1)}. \)
      \end{enumerate}
    \item[(1.9)] This problem derives an asymptotic approximation for the
      Stieltjes function, defined as
      \[
        S(\varepsilon) = \int_0^\infty \frac{e^{-t}}{1 + \varepsilon t} \,
        dt.
      \]
      \begin{enumerate}
        \item Find the first three terms in the expansion of the
          integrand for small $\varepsilon$ and explain why this requires
          that $t \ll 1/\varepsilon$.
        \item Split the integral into the sum of an integral over $0
          < t < \delta$ and one over $\delta < t < \infty$, where $1 \ll
          \delta \ll 1/\varepsilon$. Explain why the second integral is
          bounded by $e^{-\delta}$, and use your expansion in part (a) to
          find an approximation for the first integral. From this derive
          the following approximation:
          \[
            S(\varepsilon) \sim 1 - \varepsilon + 2 \varepsilon^2 + \cdots.
          \]
      \end{enumerate}
    \item[(1.16)] This problem derives asymptotic approximations for the
      complete elliptic integral, defined as
      \[
        K(x) = \int_0^{\pi/2} \frac{ds}{\sqrt{1 - x \sin^2 s}}.
      \]
      It is assumed that $0 < x < 1$.
      \begin{enumerate}
          \item Show that, for $x$ close to zero, 
            \[
              K \sim \frac{\pi}{2} \left(1 + \frac{1}{4}x\right).
            \]
          \item Show that, for $x$ close to one, 
            \[
              K \sim -\frac{1}{2} \ln(1 - x).
            \]
          \item Show that, for $x$ close to one,
            \[
              K \sim -\frac{1}{2} \left[ 1 + \frac{1}{4}(1 - x) \right]
              \ln(1 - x).
            \]
      \end{enumerate}
  \end{enumerate}
\end{problem}

\begin{problem}
  Use integration by parts to find asymptotic expansion for large $x$ of
    the following integrals:
  \begin{enumerate}
      \item $\displaystyle \int_1^\infty e^{-xt}/t^n \, dt, \quad n \in
        \mathbb{N}.$
      \item $\displaystyle \int_0^1 e^{ixt} t^{-1/2} \, dt.$
  \end{enumerate}
\end{problem}

\begin{problem}
  This exercise examines Laplace’s method carefully as applied to
  \[
  I_n(x) = \int_0^\pi e^{x \cos t} \cos(nt) \, dt, \quad x \to \infty.
  \]
  \begin{enumerate}
      \item Show that
      \[
      I(x) \sim \int_0^\epsilon e^{x \cos t} \cos(nt) \, dt
      \]
      for any fixed $\epsilon > 0$.
      \item Prove that
      \[
        \int_0^\epsilon e^{x \cos t} \cos(nt) \, dt \sim \int_0^\epsilon e^{x(1
        - t^2)} \, dt, \quad \epsilon > 0
      \]
      by breaking up the range of integration $[0, \epsilon)$ into $[0,
      x^{-\alpha}]$ and $[x^{-\alpha}, \epsilon]$ with $\alpha \in (1/4,
      1/2)$ and showing
      \[
        \cos(nt) e^{x \cos t} \sim e^{x(1 - t^2/2)} \quad \text{uniformly for
        all } t \in [0, x^{-\alpha}],
      \]
      as $x \to \infty$ (Hint: $1 - t^2/2 \leq \cos t \leq 1 - t^2/2 +
      t^4/4!$) and the integration range $[x^{-\alpha}, \epsilon]$ has an
      exponentially smaller contribution to the integral.
      \item What’s wrong with the result if you approximate $\cos t$ by 1 and
        $\cos(nt)$ by 1 for $t \in [0, \epsilon]$?
  \end{enumerate}
\end{problem}

\end{document}

