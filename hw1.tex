\documentclass[11pt]{article}

% Packages
\usepackage[utf8]{inputenc}
\usepackage{amsmath, amssymb, amsthm}
\usepackage{enumitem}
\usepackage{geometry}
\usepackage{fancyhdr}
\usepackage{mathtools}
\usepackage{tikz}
\usepackage{hyperref}
\usepackage{multicol}

% Page layout
\geometry{margin=0.75in}
\pagestyle{fancy}
\fancyhf{}
\rhead{Ian Gallagher}
\lhead{Math 207C Homework}
\rfoot{\thepage}
\setlength{\headheight}{14pt}

% Environments
\newtheoremstyle{problemstyle}
  {1em} % Space above
  {1em} % Space below
  {\normalfont} % Body font
  {} % Indent amount
  {\bfseries} % Theorem head font
  {} % Punctuation after theorem head
  {\newline} % Space after theorem head
  {} % Theorem head spec

\theoremstyle{problemstyle}
\newtheorem{problem}{Problem}

% Custom commands
\newenvironment{solution}
  {\noindent\textbf{Solution}\quad}
  {\hfill$\blacksquare$\par\vspace{1em}}

% Enumerate styles
\setlist[enumerate,1]{label=(\alph*), ref=\alph*, itemsep=0.2em, topsep=0.4em}
\setlist[enumerate,2]{label=(\roman*), ref=\roman*, itemsep=0.1em, topsep=0.2em}

% Title info
\title{Math 207C Homework \#\texttt{1}}
\author{Ian Gallagher}
\date{\today}

\begin{document}

\maketitle

\noindent Holmes Ex \#1.1, 1.3, 1.5, 1.7, 1.12, 1.18 (a)(e)(h)(p), 1.19, 1.20, 1.21

% Example Problem Format
\begin{problem}[1.1]
\mbox{} % Add empty box to force a newline for the enumerate
\begin{enumerate}
  \item What values of $\alpha$, if any, yield 
        \(
        f = O\bigl(\varepsilon^\alpha\bigr)
        \)
        as $\varepsilon \downarrow 0$ for each of the following functions?
\begin{multicols}{2}
  \begin{enumerate}
    \item $f = \sqrt{1 + \varepsilon^2}$ 
    \item $f = \varepsilon \sin(\varepsilon)$ 
    \item $f = \bigl(1 - e^\varepsilon)^{-1}$
    \item $f = \ln(1 + \varepsilon)$
    \item $f = \varepsilon \ln(\varepsilon)$
    \item $f = \sin(1 / \varepsilon)$
    \item $f = \sqrt{x + \varepsilon}\text{, where } 0 \geq x \geq 1$ \newline\mbox{}
  \end{enumerate}
\end{multicols}
\item For the functions listed in (a), what values of $\alpha$, if any, yield
      \(
      f = o\bigl(\varepsilon^\alpha\bigr)
      \)
      as $\varepsilon \downarrow 0$?
\end{enumerate}
\end{problem}

\begin{solution}
\end{solution}

\begin{problem}[1.3]
This problem establishes some of the basic properties of the order symbols, some of which are used
extensively in this book. The limit assumed here is $\varepsilon \downarrow 0$.
\begin{enumerate}
\item If $f = o(g)$ and $g = O(h)$, or if $f = O(g)$ and $g = o(h)$, then show that $f = o(h)$. Note
  that this result can be written as $o(O(h)) = O(o(h)) = o(h)$.
\item Assuming $f = O(\phi_1)$ and $g = O(\phi_2)$, show that $f + g = O(\vert \phi_1 \vert  + \vert
  \phi_2 \vert)$. Also, explain why the absolute signs are necessary. Note that this result can be
  written as $O(f) + O(g) = O(|f| + |g|)$.
\item Assuming $f = O(\phi_1)$ and $g = O(\phi_2)$, show that $fg = O(\phi_1 \phi_2)$. This result
  can be written as $O(f) O(g) = O(fg)$.
\item Show that $O(O(f)) = O(f)$.
\item Show that $O(f)o(g) = o(f)o(g) = o(fg)$.
\end{enumerate}
\end{problem}

\begin{problem}{1.5}
Suppose $f=o(\phi)$ for small $\varepsilon$, where $f$ and $\phi$ are continuous.
\begin{enumerate}
  \item Give an example to show that it is not necessarily true that
    \[ \int_0^\varepsilon f \mathrm{d} \varepsilon = o\left(\int_0^\varepsilon \phi \mathrm{d}
    \varepsilon \right).\]
  \item Show that 
    \[ \int_0^\varepsilon f \mathrm{d} \varepsilon = o\left(\int_0^\varepsilon \vert \phi \vert
    \mathrm{d} \varepsilon \right).\]
\end{enumerate}
\end{problem}

\begin{problem}[1.7]
Assuming $f \sim a_1 \varepsilon^\alpha + a_2 \varepsilon^\beta + \cdots$, find $\alpha, \beta$
(with $\alpha < \beta$) and nonzero $a_1, a_2$ for the following functions:
\begin{enumerate}
  \item $\displaystyle f = \frac{1}{1 - e^{\varepsilon}}$.
  \item $\displaystyle f = \left[1 + \frac{1}{\cos(\varepsilon)}\right]^{\frac{3}{2}}$.
  \item $\displaystyle f = 1 + \varepsilon - 2\ln\left(1 + \varepsilon\right) - \frac{1}{1 + \varepsilon}$.
  \item $\displaystyle f = \sinh\left(\sqrt{1 + \varepsilon x}\right) \text{, for } 0 < x < \infty$.
  \item $\displaystyle f = \left(1 + \varepsilon x\right)^{\frac{1}{\varepsilon}} \text{, for } 0 < x < \infty$.
  \item $\displaystyle f = \int_{0}^{\varepsilon} \sin\left(x + \varepsilon x^{2}\right)dx$.
  \item $\displaystyle f = \displaystyle \sum_{n=1}^{\infty} \left(\frac{1}{2}\right)^n 
              \sin\left(\frac{\varepsilon}{n}\right)$.
  \item $\displaystyle \int_0^\varepsilon \sin(x + \varepsilon x^2) \mathrm{d}x$.
  \item $\displaystyle f = \int_0^1 \frac{\mathrm{d}x}{\varepsilon + x(x-1)}$.
\end{enumerate}
\end{problem}

\begin{problem}[1.18]
  Find a two-term asymptotic expansion, for small $\varepsilon$, of each solution $x$ of the
  following equations:
  \begin{enumerate}
    \item $x^2 + x - \varepsilon = 0$
    \item[(e)] $\varepsilon x^3 -x + \varepsilon = 0$
    \item[(h)] $x^2 + \varepsilon \sqrt{2+x} = \cos(\varepsilon)$
    \item[(p)] $xe^{-x} = \varepsilon$
  \end{enumerate}
\end{problem}

\begin{problem}[1.19]
  This problem considers the equation $1 + \sqrt{x^2 + \varepsilon} = e^x$.
  \begin{enumerate}
    \item Explain why there is one real root for small $\varepsilon$.
    \item Find a two-term expansion of the root.
  \end{enumerate}
\end{problem}

\begin{problem}[1.20]
  In this problem you should sketch the fucntions in each equation and then use this to determine
  the number and approximate location of the real-valued solutions. With this, find a three-term
  asymptotic expansion, for small $\varepsilon$, of the nonzero solutions.
  \begin{enumerate}
    \item $x = \tanh \left(\frac{x}{\varepsilon}\right)$,
    \item $x = \tan\left(\frac{x}{\varepsilon}\right)$.
  \end{enumerate}
\end{problem}

\begin{problem}[1.21]
  To determine the natural frequencies of an elastic string, one is faced with solving the equation
  $\tan(\lambda) = \lambda$.
  \begin{enumerate}
    \item After sketching the two functions in this equation on the same graph explain why there is
      an infinite number of solutions.
    \item To find an asymptotic expansion of the large solutions of the equation, assume that
      $\lambda \sim \varepsilon^{-\alpha}(\lambda_0 + \varepsilon^\beta\lambda_1$. Find
      $\varepsilon, \alpha, \beta, \lambda_0, \lambda_1$ (note that $\lambda_0$ and $\lambda_1$ are
      nonzero and $\beta > 0$).
  \end{enumerate}
\end{problem}

\end{document}

