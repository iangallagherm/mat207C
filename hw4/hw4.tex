\documentclass[12pt]{amsart}
\usepackage{amsmath, amssymb, amsthm}
\usepackage{enumitem}
\usepackage{geometry}
\geometry{margin=1in}
\usepackage{fancyhdr}
\usepackage{graphicx}
\usepackage{mathtools}

\setlength{\parskip}{0ex}
\setlength{\parindent}{0pt}
\setlength{\headheight}{14pt}
\setlength{\footskip}{14pt}
\pagestyle{fancy}
\fancyhf{}
\rhead{MAT 207C}
\lhead{Homework 4}
\rfoot{\thepage}

\title{MAT 207C: Homework 4}
\author{Ian Gallagher}

\begin{document}

\maketitle

\section*{Holmes 2.2: Matched Asymptotics Introduction}

\subsection*{Problem 2.2}
Find a composite expansion of the solution of the following problems:
\begin{enumerate}[label=(\alph*)]
  \setcounter{enumi}{2}
  \item $\varepsilon y'' + y(y' + 3) = 0$ for $0 < x < 1$, with $y(0) = 1$ and
    $y(1) = 1$.

  \item $\varepsilon y'' = f(x) - y'$ for $0 < x < 1$, with $y(0) = 0$ and $y(1)
    = 1$. Also, $f(x)$ is continuous.
\end{enumerate}

\subsection*{Problem 2.3}
Consider the problem
\[
\varepsilon^2 y'' + a y' = x^2 \quad \text{for } 0 < x < 1,
\]
with $y'(0) = \lambda$, $y(1) = 2$, where $a$ and $\lambda$ are positive constants.
\begin{enumerate}[label=(\alph*)]
  \item Find a first-term composite expansion for the solution. Explain why the
    approximation does not depend on $\lambda$.
\end{enumerate}

\subsection*{Problem 2.4}
A small parameter multiplying the highest derivative does not guarantee that the
solution will have a boundary layer for small values of $\varepsilon$. As
demonstrated in this problem, this can be due to the form of the differential
equation or the particular boundary conditions used in the problem.
\begin{enumerate}[label=(\alph*)]
  \setcounter{enumi}{1}
  \item Consider the equation $\varepsilon^2 y'' - x y' = 0$ for $0 < x < 1$.
    From the exact solution, show that there is no boundary layer if the
    boundary conditions are $y(0) = y(1) = 2$, while there is a boundary layer
    if the boundary conditions are $y(0) = 1$ and $y(1) = 2$.
\end{enumerate}

\subsection*{Problem 2.9}
Consider the problem
\[
\varepsilon y'' + p(x)y' + q(x)y = f(x), \quad 0 < x < 1,
\]
with $y(0) = \alpha$ and $y(1) = \beta$. Assume $p(x), q(x)$, and $f(x)$ are
continuous and $p(x) > 0$ for $0 \le x \le 1$.
\begin{enumerate}[label=(\alph*)]
  \item In the case where $f = 0$, show that
  \[
  y \sim \beta \exp\left( \int_x^1 \frac{q(s)}{p(s)}\, ds \right) + \left[
  \alpha - \beta \exp\left( \int_0^1 \frac{q(s)}{p(s)}\, ds \right) \right]
  h(x),
  \]
  where $h(x) = e^{-p(0)x/\varepsilon}$.
\end{enumerate}

\subsection*{Problem 2.10}
Consider the problem
\[
\varepsilon y'' + 6\sqrt{x}y' - 3y = -3, \quad 0 < x < 1,
\]
with $y(0) = 0$ and $y(1) = 3$.
\begin{enumerate}[label=(\alph*)]
  \item Find a composite expansion of the problem.
\end{enumerate}

\section*{Holmes 2.3: Boundary Layers}

\subsection*{Problem 2.15}
Find a composite expansion of the solution of the following problems
and sketch the solution:
\begin{enumerate}[label=(\alph*)]
  \setcounter{enumi}{1}
  \item $\varepsilon y'' - y' + y^2 = 1$ for $0 < x < 1$, with $y(0) = 1/3$,
    $y(1) = 1$.
  
  \setcounter{enumi}{4}
  \item $\varepsilon y'' - y(y' + 1) = 0$ for $0 < x < 1$, with $y(0) = 3$,
    $y(1) = 3$.
\end{enumerate}

\section*{Holmes 2.5: Interior Layers}

\subsection*{Problem 2.32}
Find a first-term expansion of the solution of each of the following
problems. It should not be unexpected that for the nonlinear problems the
solutions are defined implicitly or that the transition layer contains an
undetermined constant.
\begin{enumerate}[label=(\alph*)]
  \item $\varepsilon y'' = -\left(x^2 - \tfrac{1}{4} \right)y'$ for $0 < x < 1$,
    with $y(0) = 1$ and $y(1) = -1$.

  \setcounter{enumi}{5}
  \item $\varepsilon y'' + y(y' + 3) = 0$ for $0 < x < 1$, with $y(0) = -1$ and
    $y(1) = 2$.
\end{enumerate}

\subsection*{Problem 2.33}
Consider the problem
\[
\varepsilon y'' = y y' \quad \text{for } 0 < x < 1,
\]
with $y(0) = a$ and $y(1) = -a$, where $a > 0$.
\begin{enumerate}[label=(\alph*)]
  \setcounter{enumi}{1}
  \item Find a composite expansion of the solution.
\end{enumerate}

\end{document}
